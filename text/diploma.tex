\documentclass[12pt, specialist, subf, substylefile = spbu.rtx]{disser}
\usepackage[a4paper, includefoot,
            left=3cm, right=1.5cm,
            top=2cm, bottom=2cm,
            headsep=1cm, footskip=1cm]{geometry}

\usepackage[T2A]{fontenc}
\usepackage[utf8]{inputenc}
\usepackage[english, russian]{babel}
\usepackage{moreverb}
\usepackage{array}

\setcounter{tocdepth}{2}

\graphicspath{{fig/}}
\begin{document}
\institution{%
	Санкт-Петербургский государственный университет\\
	Кафедра системного программирования
}
\title{Отчет по преддипломная работа}
\topic{\normalfont\scshape %
<<Поиск визуально схожих изображений в неорганизованных коллекциях>>}
\author{Смирнов Кирилл Вадимович}
\sa {А.\,А.~Пименов}
\sastatus{доцент}

\city{Санкт-Петербург}
\date{2018}
\maketitle

\tableofcontents
\section{Введение}
Задача поиска для исходного изображения набора изображених визуально на него похожих (image retrieval) является фундаментальной в области компьютерного зрения. Среди применений этой задачи можно отметить определение местоположения на основе виззуальной схожести сцен, кластарезация сцен и детектирование циклов.
\par Наиболее популярные подходы для решения данной задачи можно разделить на те, которые испльзуют машинное обучение, и те, которые используют мешок слов (bag of words). 
\par В силу актуальности задачи существуют несколько доступных наборов данных. Наиболее удобные являются те, которые содержат в себе полную информацию о движении. Это значит, что авторами предоставляются трехмерные координаты и углы поворотов вокруг каждой из трех перпендикулярной оси (6-Dof). Примерами таких наборов данных являются KITTi и The New College.

\newpage
\section{Постановка задачи}
Целью данный работы является реализация системы поиска схожих изображений в реальном времени на основании проведенных изысканий. Для ее достижения сформулированы следующие задаччи:
\begin{itemize}
    \item Произвести анализ доступных наборов данных
    \item Произвести разметку наборов данных по геометрической близости
    \item Реализовать различные метрики схожести для сравнения их эффективности
    \item Произвести эксперементы с реализованными метриками на доступных наборах данных
    \item Реализовать систему для локализации в настоящем времени
\end{itemize}

\newpage
\section{Обзор решений}

\newpage
\section{Результаты}
За время преддипломной практики было произведено исследование предметной области - изучены основные научные труды. Была произведена работа по анализу доступных наборов данных: выявлены сложности, связанные с большим объемом данных, что делает задачу их хранения нетривиальной, найдены особенности, связанные с многоуровневыми развязками, как следствие для близких географических координат возможно большое визуальное отличие. Данные сложности были решены для двух наборов данных: KITTi и The New College. Для них были реализованы алгоритмы, производящие разметку данных, и получены разбиения на геометрически близкие изображения.

\end{document}

